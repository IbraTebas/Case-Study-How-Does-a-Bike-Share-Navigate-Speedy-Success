% Options for packages loaded elsewhere
\PassOptionsToPackage{unicode}{hyperref}
\PassOptionsToPackage{hyphens}{url}
%
\documentclass[
]{article}
\usepackage{amsmath,amssymb}
\usepackage{lmodern}
\usepackage{iftex}
\ifPDFTeX
  \usepackage[T1]{fontenc}
  \usepackage[utf8]{inputenc}
  \usepackage{textcomp} % provide euro and other symbols
\else % if luatex or xetex
  \usepackage{unicode-math}
  \defaultfontfeatures{Scale=MatchLowercase}
  \defaultfontfeatures[\rmfamily]{Ligatures=TeX,Scale=1}
\fi
% Use upquote if available, for straight quotes in verbatim environments
\IfFileExists{upquote.sty}{\usepackage{upquote}}{}
\IfFileExists{microtype.sty}{% use microtype if available
  \usepackage[]{microtype}
  \UseMicrotypeSet[protrusion]{basicmath} % disable protrusion for tt fonts
}{}
\makeatletter
\@ifundefined{KOMAClassName}{% if non-KOMA class
  \IfFileExists{parskip.sty}{%
    \usepackage{parskip}
  }{% else
    \setlength{\parindent}{0pt}
    \setlength{\parskip}{6pt plus 2pt minus 1pt}}
}{% if KOMA class
  \KOMAoptions{parskip=half}}
\makeatother
\usepackage{xcolor}
\IfFileExists{xurl.sty}{\usepackage{xurl}}{} % add URL line breaks if available
\IfFileExists{bookmark.sty}{\usepackage{bookmark}}{\usepackage{hyperref}}
\hypersetup{
  hidelinks,
  pdfcreator={LaTeX via pandoc}}
\urlstyle{same} % disable monospaced font for URLs
\usepackage[margin=1in]{geometry}
\usepackage{color}
\usepackage{fancyvrb}
\newcommand{\VerbBar}{|}
\newcommand{\VERB}{\Verb[commandchars=\\\{\}]}
\DefineVerbatimEnvironment{Highlighting}{Verbatim}{commandchars=\\\{\}}
% Add ',fontsize=\small' for more characters per line
\usepackage{framed}
\definecolor{shadecolor}{RGB}{248,248,248}
\newenvironment{Shaded}{\begin{snugshade}}{\end{snugshade}}
\newcommand{\AlertTok}[1]{\textcolor[rgb]{0.94,0.16,0.16}{#1}}
\newcommand{\AnnotationTok}[1]{\textcolor[rgb]{0.56,0.35,0.01}{\textbf{\textit{#1}}}}
\newcommand{\AttributeTok}[1]{\textcolor[rgb]{0.77,0.63,0.00}{#1}}
\newcommand{\BaseNTok}[1]{\textcolor[rgb]{0.00,0.00,0.81}{#1}}
\newcommand{\BuiltInTok}[1]{#1}
\newcommand{\CharTok}[1]{\textcolor[rgb]{0.31,0.60,0.02}{#1}}
\newcommand{\CommentTok}[1]{\textcolor[rgb]{0.56,0.35,0.01}{\textit{#1}}}
\newcommand{\CommentVarTok}[1]{\textcolor[rgb]{0.56,0.35,0.01}{\textbf{\textit{#1}}}}
\newcommand{\ConstantTok}[1]{\textcolor[rgb]{0.00,0.00,0.00}{#1}}
\newcommand{\ControlFlowTok}[1]{\textcolor[rgb]{0.13,0.29,0.53}{\textbf{#1}}}
\newcommand{\DataTypeTok}[1]{\textcolor[rgb]{0.13,0.29,0.53}{#1}}
\newcommand{\DecValTok}[1]{\textcolor[rgb]{0.00,0.00,0.81}{#1}}
\newcommand{\DocumentationTok}[1]{\textcolor[rgb]{0.56,0.35,0.01}{\textbf{\textit{#1}}}}
\newcommand{\ErrorTok}[1]{\textcolor[rgb]{0.64,0.00,0.00}{\textbf{#1}}}
\newcommand{\ExtensionTok}[1]{#1}
\newcommand{\FloatTok}[1]{\textcolor[rgb]{0.00,0.00,0.81}{#1}}
\newcommand{\FunctionTok}[1]{\textcolor[rgb]{0.00,0.00,0.00}{#1}}
\newcommand{\ImportTok}[1]{#1}
\newcommand{\InformationTok}[1]{\textcolor[rgb]{0.56,0.35,0.01}{\textbf{\textit{#1}}}}
\newcommand{\KeywordTok}[1]{\textcolor[rgb]{0.13,0.29,0.53}{\textbf{#1}}}
\newcommand{\NormalTok}[1]{#1}
\newcommand{\OperatorTok}[1]{\textcolor[rgb]{0.81,0.36,0.00}{\textbf{#1}}}
\newcommand{\OtherTok}[1]{\textcolor[rgb]{0.56,0.35,0.01}{#1}}
\newcommand{\PreprocessorTok}[1]{\textcolor[rgb]{0.56,0.35,0.01}{\textit{#1}}}
\newcommand{\RegionMarkerTok}[1]{#1}
\newcommand{\SpecialCharTok}[1]{\textcolor[rgb]{0.00,0.00,0.00}{#1}}
\newcommand{\SpecialStringTok}[1]{\textcolor[rgb]{0.31,0.60,0.02}{#1}}
\newcommand{\StringTok}[1]{\textcolor[rgb]{0.31,0.60,0.02}{#1}}
\newcommand{\VariableTok}[1]{\textcolor[rgb]{0.00,0.00,0.00}{#1}}
\newcommand{\VerbatimStringTok}[1]{\textcolor[rgb]{0.31,0.60,0.02}{#1}}
\newcommand{\WarningTok}[1]{\textcolor[rgb]{0.56,0.35,0.01}{\textbf{\textit{#1}}}}
\usepackage{graphicx}
\makeatletter
\def\maxwidth{\ifdim\Gin@nat@width>\linewidth\linewidth\else\Gin@nat@width\fi}
\def\maxheight{\ifdim\Gin@nat@height>\textheight\textheight\else\Gin@nat@height\fi}
\makeatother
% Scale images if necessary, so that they will not overflow the page
% margins by default, and it is still possible to overwrite the defaults
% using explicit options in \includegraphics[width, height, ...]{}
\setkeys{Gin}{width=\maxwidth,height=\maxheight,keepaspectratio}
% Set default figure placement to htbp
\makeatletter
\def\fps@figure{htbp}
\makeatother
\setlength{\emergencystretch}{3em} % prevent overfull lines
\providecommand{\tightlist}{%
  \setlength{\itemsep}{0pt}\setlength{\parskip}{0pt}}
\setcounter{secnumdepth}{-\maxdimen} % remove section numbering
\ifLuaTeX
  \usepackage{selnolig}  % disable illegal ligatures
\fi

\author{}
\date{\vspace{-2.5em}}

\begin{document}

\hypertarget{journal-data-collection-validation-and-cleaning}{%
\section{Journal data collection, validation and
cleaning}\label{journal-data-collection-validation-and-cleaning}}

\hypertarget{data-collection}{%
\paragraph{\texorpdfstring{\emph{Data
collection}}{Data collection}}\label{data-collection}}

\hypertarget{downloaded-the-data-from-httpsdivvy-tripdata.s3.amazonaws.comindex.html}{%
\subparagraph{\texorpdfstring{- Downloaded the data from
\url{https://divvy-tripdata.s3.amazonaws.com/index.html}}{- Downloaded the data from https://divvy-tripdata.s3.amazonaws.com/index.html}}\label{downloaded-the-data-from-httpsdivvy-tripdata.s3.amazonaws.comindex.html}}

\hypertarget{due-to-some-of-the-files-being-greater-in-size-than-100mb-i-used-a-gnu-general-public-license-version-3.0-gplv3-free-huge-csv-splitter-software-to-create-files-all-lesser-than-100-mb.}{%
\subparagraph{\texorpdfstring{- Due to some of the files being greater
in size than 100MB, I used a GNU General Public License version 3.0
(GPLv3) ``Free Huge CSV Splitter'' software to create files all lesser
than 100 MB.
}{- Due to some of the files being greater in size than 100MB, I used a GNU General Public License version 3.0 (GPLv3) ``Free Huge CSV Splitter'' software to create files all lesser than 100 MB.  }}\label{due-to-some-of-the-files-being-greater-in-size-than-100mb-i-used-a-gnu-general-public-license-version-3.0-gplv3-free-huge-csv-splitter-software-to-create-files-all-lesser-than-100-mb.}}

\hypertarget{data-processing}{%
\paragraph{\texorpdfstring{\emph{Data
Processing}}{Data Processing}}\label{data-processing}}

\hypertarget{i-have-obtained-17-files-in-total-i-use-the-below-script-in-python-to-add-the-headers-on-the-split-files-the-split-files-withouth-headers-ended-with-a-ii-notation}{%
\subparagraph{I have obtained 17 files in total, I use the below script
in python to add the headers on the split files, the split files
withouth headers ended with a `II'
notation:}\label{i-have-obtained-17-files-in-total-i-use-the-below-script-in-python-to-add-the-headers-on-the-split-files-the-split-files-withouth-headers-ended-with-a-ii-notation}}

import pandas as pd import os directory = `./no\_names/' tester =
pd.read\_csv(`TESTER.csv') names = list(tester.columns) for filname in
os.listdir(directory): temp = str(filname) df = pd.read\_csv(directory +
str(filname)) df.columns = names df.to\_csv(`headed\_\{\}'.format(temp),
index = False)

\hypertarget{i-have-uploaded-the-csv-files-to-a-bigquery-project.-i-have-executed-a-query-to-do-union-all-of-the-tables-and-store-the-result-in-a-csv-file-selecting-the-option-for-not-limit-in-size-on-the-save-file}{%
\paragraph{I have uploaded the csv files to a BigQuery Project. I have
executed a Query to do ``Union ALL'' of the tables and store the result
in a csv file, selecting the option for not limit in size on the save
file}\label{i-have-uploaded-the-csv-files-to-a-bigquery-project.-i-have-executed-a-query-to-do-union-all-of-the-tables-and-store-the-result-in-a-csv-file-selecting-the-option-for-not-limit-in-size-on-the-save-file}}

SELECT * FROM \texttt{tidal-tower-348213.Trips\_data.data\_2021\_04}
UNION ALL SELECT * FROM
\texttt{tidal-tower-348213.Trips\_data.data\_2021\_05} UNION ALL SELECT
* FROM \texttt{tidal-tower-348213.Trips\_data.data\_2021\_06\_I} UNION
ALL SELECT * FROM
\texttt{tidal-tower-348213.Trips\_data.data\_2021\_06\_II} UNION ALL
SELECT * FROM \texttt{tidal-tower-348213.Trips\_data.data\_2021\_07\_I}
UNION ALL SELECT * FROM
\texttt{tidal-tower-348213.Trips\_data.data\_2021\_07\_II} UNION ALL
SELECT * FROM \texttt{tidal-tower-348213.Trips\_data.data\_2021\_08\_I}
UNION ALL SELECT * FROM
\texttt{tidal-tower-348213.Trips\_data.data\_2021\_08\_II} UNION ALL
SELECT * FROM \texttt{tidal-tower-348213.Trips\_data.data\_2021\_09\_I}
UNION ALL SELECT * FROM
\texttt{tidal-tower-348213.Trips\_data.data\_2021\_09\_II} UNION ALL
SELECT * FROM \texttt{tidal-tower-348213.Trips\_data.data\_2021\_10\_I}
UNION ALL SELECT * FROM
\texttt{tidal-tower-348213.Trips\_data.data\_2021\_10\_II} UNION ALL
SELECT * FROM \texttt{tidal-tower-348213.Trips\_data.data\_2021\_11}
UNION ALL SELECT * FROM
\texttt{tidal-tower-348213.Trips\_data.data\_2021\_12} UNION ALL SELECT
* FROM \texttt{tidal-tower-348213.Trips\_data.data\_2022\_01} UNION ALL
SELECT * FROM \texttt{tidal-tower-348213.Trips\_data.data\_2022\_02}
UNION ALL SELECT * FROM
\texttt{tidal-tower-348213.Trips\_data.data\_2022\_03}

\hypertarget{data-cleaning-in-r}{%
\subsubsection{- Data Cleaning in R}\label{data-cleaning-in-r}}

\hypertarget{lets-start-by-loading-the-library-and-installing-packages-and-then-to-load-the-data-stored-on-the-.csv-file-on-the-above-query.}{%
\paragraph{Let's start by loading the library and installing packages,
and then to load the data stored on the .CSV file on the above
query.}\label{lets-start-by-loading-the-library-and-installing-packages-and-then-to-load-the-data-stored-on-the-.csv-file-on-the-above-query.}}

\begin{Shaded}
\begin{Highlighting}[]
\CommentTok{\# install.packages("leaflet")}
\CommentTok{\# install.packages(\textquotesingle{}geosphere\textquotesingle{})}
\CommentTok{\# install.packages(\textquotesingle{}matrixStats\textquotesingle{})}
\CommentTok{\# install.packages(\textquotesingle{}dplyr\textquotesingle{})}
\CommentTok{\# install.packages(\textquotesingle{}plyr\textquotesingle{})}
\CommentTok{\# install.packages(\textquotesingle{}vtable\textquotesingle{})}
\CommentTok{\# }
\FunctionTok{library}\NormalTok{(dplyr)}
\end{Highlighting}
\end{Shaded}

\begin{verbatim}
## 
## Attaching package: 'dplyr'
\end{verbatim}

\begin{verbatim}
## The following objects are masked from 'package:stats':
## 
##     filter, lag
\end{verbatim}

\begin{verbatim}
## The following objects are masked from 'package:base':
## 
##     intersect, setdiff, setequal, union
\end{verbatim}

\begin{Shaded}
\begin{Highlighting}[]
\FunctionTok{library}\NormalTok{(leaflet)}
\FunctionTok{library}\NormalTok{(tidyverse)}
\end{Highlighting}
\end{Shaded}

\begin{verbatim}
## -- Attaching packages --------------------------------------- tidyverse 1.3.1 --
\end{verbatim}

\begin{verbatim}
## v ggplot2 3.3.5     v purrr   0.3.4
## v tibble  3.1.6     v stringr 1.4.0
## v tidyr   1.2.0     v forcats 0.5.1
## v readr   2.1.2
\end{verbatim}

\begin{verbatim}
## -- Conflicts ------------------------------------------ tidyverse_conflicts() --
## x dplyr::filter() masks stats::filter()
## x dplyr::lag()    masks stats::lag()
\end{verbatim}

\begin{Shaded}
\begin{Highlighting}[]
\FunctionTok{library}\NormalTok{(geosphere)}
\FunctionTok{library}\NormalTok{(matrixStats)}
\end{Highlighting}
\end{Shaded}

\begin{verbatim}
## 
## Attaching package: 'matrixStats'
\end{verbatim}

\begin{verbatim}
## The following object is masked from 'package:dplyr':
## 
##     count
\end{verbatim}

\begin{Shaded}
\begin{Highlighting}[]
\NormalTok{data }\OtherTok{\textless{}{-}} \FunctionTok{read.csv}\NormalTok{(}\StringTok{"data\_12\_months.csv"}\NormalTok{)}
\end{Highlighting}
\end{Shaded}

\hypertarget{eliminated-duplicated-rows-dataframe-rows-count-down-from-5917935-to-5386302.}{%
\paragraph{-Eliminated duplicated rows: Dataframe rows count down from
5,917,935 to
5,386,302.}\label{eliminated-duplicated-rows-dataframe-rows-count-down-from-5917935-to-5386302.}}

\hypertarget{eliminated-rows-with-no-end-coordinates-rows-down-from-5386302-to-5381853.}{%
\paragraph{-Eliminated rows with no ``end coordinates'': rows down from
5,386,302 to
5,381,853.}\label{eliminated-rows-with-no-end-coordinates-rows-down-from-5386302-to-5381853.}}

\begin{Shaded}
\begin{Highlighting}[]
\NormalTok{data }\OtherTok{\textless{}{-}}\NormalTok{ data }\SpecialCharTok{\%\textgreater{}\%}
  \FunctionTok{distinct}\NormalTok{(ride\_id, }\AttributeTok{.keep\_all =} \ConstantTok{TRUE}\NormalTok{) }\SpecialCharTok{\%\textgreater{}\%}                       
  \FunctionTok{drop\_na}\NormalTok{(end\_lat)        }
\end{Highlighting}
\end{Shaded}

\hypertarget{extracted-date-and-time-for-start-and-end-in-different-columns-remove-started_at-and-ended_at-columns-added-a-column-with-weekday-both-numeric-monday-1-etc-and-abreviated-as-a-string}{%
\paragraph{Extracted date and time for start and end in different
columns, remove started\_at and ended\_at columns, Added a column with
weekday both numeric (Monday = 1, etc) and abreviated as a
string}\label{extracted-date-and-time-for-start-and-end-in-different-columns-remove-started_at-and-ended_at-columns-added-a-column-with-weekday-both-numeric-monday-1-etc-and-abreviated-as-a-string}}

\begin{Shaded}
\begin{Highlighting}[]
\CommentTok{\#extract date and time in separate columns at start and end}
\NormalTok{data}\SpecialCharTok{$}\NormalTok{start\_time}\OtherTok{\textless{}{-}} \FunctionTok{format}\NormalTok{(}\FunctionTok{as.POSIXct}\NormalTok{(                          }
\NormalTok{data}\SpecialCharTok{$}\NormalTok{started\_at),}\AttributeTok{format =} \StringTok{"\%H:\%M:\%S"}\NormalTok{)}
\NormalTok{data}\SpecialCharTok{$}\NormalTok{start\_date }\OtherTok{\textless{}{-}} \FunctionTok{as.Date}\NormalTok{ (data}\SpecialCharTok{$}\NormalTok{started\_at)}

\NormalTok{data}\SpecialCharTok{$}\NormalTok{end\_time}\OtherTok{\textless{}{-}} \FunctionTok{format}\NormalTok{(}\FunctionTok{as.POSIXct}\NormalTok{(}
\NormalTok{data}\SpecialCharTok{$}\NormalTok{ended\_at),}\AttributeTok{format =} \StringTok{"\%H:\%M:\%S"}\NormalTok{)}
\NormalTok{data}\SpecialCharTok{$}\NormalTok{end\_date }\OtherTok{\textless{}{-}} \FunctionTok{as.Date}\NormalTok{ (data}\SpecialCharTok{$}\NormalTok{ended\_at)}

\CommentTok{\#add the day of the week                                                                                         }
\NormalTok{data}\SpecialCharTok{$}\NormalTok{weekday\_started }\OtherTok{=} \FunctionTok{format}\NormalTok{(data}\SpecialCharTok{$}\NormalTok{start\_date, }\AttributeTok{format =} \StringTok{"\%a"}\NormalTok{)                                }\CommentTok{\#*as a string abbreviated}
\NormalTok{data}\SpecialCharTok{$}\NormalTok{weekday\_started\_int }\OtherTok{=} \FunctionTok{as.numeric}\NormalTok{(}\FunctionTok{format}\NormalTok{(data}\SpecialCharTok{$}\NormalTok{start\_date, }\AttributeTok{format =} \StringTok{"\%u"}\NormalTok{))                }\CommentTok{\#* as an integer, Monday = 1   }
\NormalTok{data}\SpecialCharTok{$}\NormalTok{weekday\_ended }\OtherTok{=} \FunctionTok{format}\NormalTok{(data}\SpecialCharTok{$}\NormalTok{end\_date, }\AttributeTok{format =} \StringTok{"\%a"}\NormalTok{) }
\NormalTok{data}\SpecialCharTok{$}\NormalTok{weekday\_ended\_int }\OtherTok{=} \FunctionTok{as.numeric}\NormalTok{(}\FunctionTok{format}\NormalTok{(data}\SpecialCharTok{$}\NormalTok{end\_date, }\AttributeTok{format =} \StringTok{"\%u"}\NormalTok{))                    }\CommentTok{\#idem}

\CommentTok{\#calculate duration}
\NormalTok{data}\SpecialCharTok{$}\NormalTok{duration\_trip\_in\_mins }\OtherTok{\textless{}{-}} \FunctionTok{difftime}\NormalTok{(data}\SpecialCharTok{$}\NormalTok{ended\_at, data}\SpecialCharTok{$}\NormalTok{started\_at, }\AttributeTok{units =} \StringTok{"mins"}\NormalTok{) }\CommentTok{\#let\textquotesingle{}s check some stats}
\FunctionTok{mean}\NormalTok{(data}\SpecialCharTok{$}\NormalTok{duration\_trip\_in\_mins)}
\end{Highlighting}
\end{Shaded}

\begin{verbatim}
## Time difference of 20.31966 mins
\end{verbatim}

\begin{Shaded}
\begin{Highlighting}[]
\FunctionTok{max}\NormalTok{(data}\SpecialCharTok{$}\NormalTok{duration\_trip\_in\_mins)}
\end{Highlighting}
\end{Shaded}

\begin{verbatim}
## Time difference of 55944.15 mins
\end{verbatim}

\begin{Shaded}
\begin{Highlighting}[]
\FunctionTok{min}\NormalTok{(data}\SpecialCharTok{$}\NormalTok{duration\_trip\_in\_mins)}
\end{Highlighting}
\end{Shaded}

\begin{verbatim}
## Time difference of -58.03333 mins
\end{verbatim}

\begin{Shaded}
\begin{Highlighting}[]
\CommentTok{\#removed the date columns with combination of day and time}
\NormalTok{data }\OtherTok{\textless{}{-}} \FunctionTok{subset}\NormalTok{(data, }\AttributeTok{select =} \SpecialCharTok{{-}}\FunctionTok{c}\NormalTok{(started\_at, ended\_at))   }
\end{Highlighting}
\end{Shaded}

\hypertarget{it-is-observable-that-the-data-duration-seems-to-have-outliers-some-negative-differences-and-values-that-are-not-coherent-in-terms-of-trip-duration-for-this-kind-of-servcie-this-woudl-be-process-later-on-to-make-sure-the-data-is-clean-and-consistent.}{%
\paragraph{It is observable that the data duration seems to have
outliers, some negative differences and values that are not coherent in
terms of trip duration for this kind of servcie, this woudl be process
later on to make sure the data is clean and
consistent.}\label{it-is-observable-that-the-data-duration-seems-to-have-outliers-some-negative-differences-and-values-that-are-not-coherent-in-terms-of-trip-duration-for-this-kind-of-servcie-this-woudl-be-process-later-on-to-make-sure-the-data-is-clean-and-consistent.}}

\hypertarget{creation-of-the-haversine-function-to-calculate-the-distances-in-km-between-two-points-given-by-their-geographical-coordinates.}{%
\paragraph{Creation of the `haversine' function, to calculate the
distances in Km between two points given by their geographical
coordinates.}\label{creation-of-the-haversine-function-to-calculate-the-distances-in-km-between-two-points-given-by-their-geographical-coordinates.}}

\begin{Shaded}
\begin{Highlighting}[]
\NormalTok{haversine}\OtherTok{\textless{}{-}} \ControlFlowTok{function}\NormalTok{(long1, lat1, long2, lat2) \{}
  
  \FunctionTok{stopifnot}\NormalTok{(}\FunctionTok{is.numeric}\NormalTok{(long1),}
            \FunctionTok{is.numeric}\NormalTok{(lat1),}
            \FunctionTok{is.numeric}\NormalTok{(long2),}
            \FunctionTok{is.numeric}\NormalTok{(lat2),}
\NormalTok{            long1 }\SpecialCharTok{\textgreater{}} \SpecialCharTok{{-}}\DecValTok{180}\NormalTok{,}
\NormalTok{            long1 }\SpecialCharTok{\textless{}} \DecValTok{180}\NormalTok{,}
\NormalTok{            lat1 }\SpecialCharTok{\textgreater{}} \SpecialCharTok{{-}}\DecValTok{180}\NormalTok{,}
\NormalTok{            lat1 }\SpecialCharTok{\textless{}} \DecValTok{180}\NormalTok{,}
\NormalTok{            long2 }\SpecialCharTok{\textgreater{}} \SpecialCharTok{{-}}\DecValTok{180}\NormalTok{,}
\NormalTok{            long2 }\SpecialCharTok{\textless{}} \DecValTok{180}\NormalTok{,}
\NormalTok{            lat2 }\SpecialCharTok{\textgreater{}} \SpecialCharTok{{-}}\DecValTok{180}\NormalTok{,}
\NormalTok{            lat2 }\SpecialCharTok{\textless{}} \DecValTok{180}
\NormalTok{  )}
  
\NormalTok{  long1 }\OtherTok{\textless{}{-}}\NormalTok{ long1}\SpecialCharTok{*}\NormalTok{pi}\SpecialCharTok{/}\DecValTok{180}
\NormalTok{  lat1 }\OtherTok{\textless{}{-}}\NormalTok{ lat1}\SpecialCharTok{*}\NormalTok{pi}\SpecialCharTok{/}\DecValTok{180}
\NormalTok{  long2 }\OtherTok{\textless{}{-}}\NormalTok{ long2}\SpecialCharTok{*}\NormalTok{pi}\SpecialCharTok{/}\DecValTok{180}
\NormalTok{  lat2 }\OtherTok{\textless{}{-}}\NormalTok{ lat2}\SpecialCharTok{*}\NormalTok{pi}\SpecialCharTok{/}\DecValTok{180}
  
\NormalTok{  R }\OtherTok{\textless{}{-}} \DecValTok{6371} \CommentTok{\# Earth mean radius [km]}
\NormalTok{  delta.long }\OtherTok{\textless{}{-}}\NormalTok{ (long2 }\SpecialCharTok{{-}}\NormalTok{ long1)}
\NormalTok{  delta.lat }\OtherTok{\textless{}{-}}\NormalTok{ (lat2 }\SpecialCharTok{{-}}\NormalTok{ lat1)}
\NormalTok{  a }\OtherTok{\textless{}{-}} \FunctionTok{sin}\NormalTok{(delta.lat}\SpecialCharTok{/}\DecValTok{2}\NormalTok{)}\SpecialCharTok{\^{}}\DecValTok{2} \SpecialCharTok{+} \FunctionTok{cos}\NormalTok{(lat1) }\SpecialCharTok{*} \FunctionTok{cos}\NormalTok{(lat2) }\SpecialCharTok{*} \FunctionTok{sin}\NormalTok{(delta.long}\SpecialCharTok{/}\DecValTok{2}\NormalTok{)}\SpecialCharTok{\^{}}\DecValTok{2}
\NormalTok{  c }\OtherTok{\textless{}{-}} \DecValTok{2} \SpecialCharTok{*} \FunctionTok{asin}\NormalTok{(}\FunctionTok{min}\NormalTok{(}\DecValTok{1}\NormalTok{,}\FunctionTok{sqrt}\NormalTok{(a)))}
\NormalTok{  d }\OtherTok{=}\NormalTok{ R }\SpecialCharTok{*}\NormalTok{ c}
  \FunctionTok{return}\NormalTok{(d) }\CommentTok{\# Distance in km}
\NormalTok{\}}
\end{Highlighting}
\end{Shaded}

\hypertarget{adding-a-column-for-distance-in-km-per-trip.}{%
\paragraph{Adding a column for distance in km per
trip.}\label{adding-a-column-for-distance-in-km-per-trip.}}

\begin{Shaded}
\begin{Highlighting}[]
\NormalTok{data}\SpecialCharTok{$}\NormalTok{distance\_in\_km }\OtherTok{\textless{}{-}} \FunctionTok{with}\NormalTok{(data, }\FunctionTok{mapply}\NormalTok{(haversine, }\AttributeTok{lat1=}\NormalTok{data}\SpecialCharTok{$}\NormalTok{start\_lat, }\AttributeTok{long1=}\NormalTok{data}\SpecialCharTok{$}\NormalTok{start\_lng, }\AttributeTok{lat2=}\NormalTok{data}\SpecialCharTok{$}\NormalTok{end\_lat, }\AttributeTok{long2=}\NormalTok{data}\SpecialCharTok{$}\NormalTok{end\_lng))}
\end{Highlighting}
\end{Shaded}

\hypertarget{lets-summarise-the-numerical-variables-to-explore-for-outliers-based-on-this-lets-filtter-the-data-to-observe-this-outliers.}{%
\paragraph{Let's summarise the numerical variables to explore for
outliers, based on this, lets filtter the data to observe this
outliers.}\label{lets-summarise-the-numerical-variables-to-explore-for-outliers-based-on-this-lets-filtter-the-data-to-observe-this-outliers.}}

\begin{Shaded}
\begin{Highlighting}[]
\NormalTok{data }\SpecialCharTok{\%\textgreater{}\%} \FunctionTok{summarise\_if}\NormalTok{(is.numeric, max)}
\end{Highlighting}
\end{Shaded}

\begin{verbatim}
##         X start_lat start_lng  end_lat end_lng weekday_started_int
## 1 5917935  45.63503 -73.79648 42.16812  -87.49                   7
##   weekday_ended_int distance_in_km
## 1                 7       1189.522
\end{verbatim}

\begin{Shaded}
\begin{Highlighting}[]
\NormalTok{data }\SpecialCharTok{\%\textgreater{}\%}\FunctionTok{summarise\_if}\NormalTok{(is.numeric, min)}
\end{Highlighting}
\end{Shaded}

\begin{verbatim}
##   X start_lat start_lng end_lat end_lng weekday_started_int weekday_ended_int
## 1 1     41.64    -87.84   41.39  -88.97                   1                 1
##   distance_in_km
## 1              0
\end{verbatim}

\begin{Shaded}
\begin{Highlighting}[]
\NormalTok{outliers\_distance\_max }\OtherTok{\textless{}{-}}\NormalTok{ data }\SpecialCharTok{\%\textgreater{}\%} \FunctionTok{filter}\NormalTok{(distance\_in\_km }\SpecialCharTok{\textgreater{}} \DecValTok{1000}\NormalTok{)}
\NormalTok{outliers\_distance\_and\_duration }\OtherTok{\textless{}{-}}\NormalTok{ data }\SpecialCharTok{\%\textgreater{}\%} \FunctionTok{filter}\NormalTok{(distance\_in\_km }\SpecialCharTok{==} \DecValTok{0} \SpecialCharTok{\&}\NormalTok{ duration\_trip\_in\_mins }\SpecialCharTok{\textless{}} \DecValTok{5}\NormalTok{)}\CommentTok{\# we will remove the trips that the bike is docked in the same station and less than 5 min have passed. Candidate to abandon trip.}
\NormalTok{Outlier\_graph }\OtherTok{\textless{}{-}}\NormalTok{data }\SpecialCharTok{\%\textgreater{}\%} \FunctionTok{filter}\NormalTok{(ride\_id }\SpecialCharTok{==} \StringTok{"9F438AD0AB380E3F"}\NormalTok{ )  }\CommentTok{\#ride id detected as an outlier on Tableau when checking end lat and lng.}
\NormalTok{outlier\_time\_more\_than\_5\_hrs }\OtherTok{\textless{}{-}}\NormalTok{ data }\SpecialCharTok{\%\textgreater{}\%} \FunctionTok{filter}\NormalTok{(duration\_trip\_in\_mins }\SpecialCharTok{\textgreater{}} \DecValTok{300}\NormalTok{)  }\CommentTok{\#check trip duration greater than 5 hours}
\NormalTok{outlier\_time\_null\_time\_or\_negative }\OtherTok{\textless{}{-}}\NormalTok{ data }\SpecialCharTok{\%\textgreater{}\%} \FunctionTok{filter}\NormalTok{(duration\_trip\_in\_mins }\SpecialCharTok{\textless{}} \DecValTok{0}\NormalTok{ ) }\CommentTok{\# check negative duration }
\end{Highlighting}
\end{Shaded}

\hypertarget{removal-of-outliers}{%
\paragraph{Removal of outliers}\label{removal-of-outliers}}

\begin{Shaded}
\begin{Highlighting}[]
\CommentTok{\#Removal of the correspondent rows for the outliers}
\NormalTok{data }\OtherTok{\textless{}{-}} \FunctionTok{subset}\NormalTok{(data, distance\_in\_km }\SpecialCharTok{!=} \DecValTok{0} \SpecialCharTok{|}\NormalTok{ duration\_trip\_in\_mins }\SpecialCharTok{\textgreater{}} \DecValTok{4}\NormalTok{)     }\CommentTok{\#removal of 131855 rows}
\NormalTok{data }\OtherTok{\textless{}{-}} \FunctionTok{subset}\NormalTok{(data, distance\_in\_km }\SpecialCharTok{\textless{}} \DecValTok{1000}\NormalTok{)  }\CommentTok{\#distance more than 1000KM, tested max and it went down to 114.3836}
\NormalTok{data }\OtherTok{\textless{}{-}} \FunctionTok{subset}\NormalTok{(data, duration\_trip\_in\_mins }\SpecialCharTok{\textless{}=} \DecValTok{300} \SpecialCharTok{\&}\NormalTok{ duration\_trip\_in\_mins }\SpecialCharTok{\textgreater{}=}\DecValTok{0}\NormalTok{) }\CommentTok{\#removal of  9419 rows with more of 5 hours duration, 1161 of them more than a day.}
\NormalTok{data }\OtherTok{\textless{}{-}} \FunctionTok{subset}\NormalTok{(data, ride\_id }\SpecialCharTok{!=} \StringTok{"9F438AD0AB380E3F"}\NormalTok{)}
\end{Highlighting}
\end{Shaded}

\hypertarget{after-observing-the-rows-with-nan-values-on-the-end-or-start-station-data-it-was-detected-that-about-13-of-the-data-was-affected-at-this-pointi-analysed-the-diference-on-indicators-after-and-before-of-the-drop-of-this-nan-the-conlcusion-was-that-the-drop-of-this-info-only-affected-statistically-the-data-in-terms-of-the-type-of-bike-used-which-is-not-a-variable-relevant-for-this-study-therefore-i-dicided-to-drop-the-nan-values.}{%
\paragraph{After observing the rows with NaN values on the end or start
station data, it was detected that about 13\% of the data was affected,
at this pointI analysed the diference on indicators after and before of
the drop of this NaN, the conlcusion was that the drop of this info only
affected statistically the data in terms of the type of bike used, which
is not a variable relevant for this study, therefore I dicided to drop
the NaN
values.}\label{after-observing-the-rows-with-nan-values-on-the-end-or-start-station-data-it-was-detected-that-about-13-of-the-data-was-affected-at-this-pointi-analysed-the-diference-on-indicators-after-and-before-of-the-drop-of-this-nan-the-conlcusion-was-that-the-drop-of-this-info-only-affected-statistically-the-data-in-terms-of-the-type-of-bike-used-which-is-not-a-variable-relevant-for-this-study-therefore-i-dicided-to-drop-the-nan-values.}}

! {[}Alt text{]} (/Stats\_data\_with\_NaN\_stations.png) ! {[}Alt
text{]} (/stats\_data\_without\_NaN.png)

\end{document}
